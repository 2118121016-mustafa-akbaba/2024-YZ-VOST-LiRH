


Bu proje, bir kitap yönetim paneli geliştirmeyi amaçlayan bir uygulamadır.Kitap yönetim paneli, sadece kitap ekleme, silme, listeleme, güncelleme, arama ve filtreleme gibi temel işlevleri sağlamakla kalmayıp aynı zamanda kullanıcı dostu bir arayüzle donatılarak kitap koleksiyonlarını etkili bir şekilde yönetmeyi hedefliyorum.Projedeki veri seti, kitapların yanı sıra yazar bilgileri,fiyatlar ve kategoriler gibi çeşitli parametreleri içerecektir.Aynı zamanda bu proje 4 katmanlı bir mimari yapısana sahip bir çalışmadır.

Projede client katmanında kullanıcı etkileşimini sağlamak amacıyla JavaScript, HTML, ve CSS dilleri kullanılacaktır. Server katmanında C Sharp programlama dili kullanırken aynı zamanda .NET CORE frameworkü ve ASP.NET MVC modelinden yararlanacağım.


Database katmanında ise verileri depolamak için PostgreSQL veritabanı yönetim sistemini kullanacağım.Bu teknoloji yığını, projenin ihtiyaçlarına uygun bir şekilde seçilmiştir ve kullanıcıya kitap yönetim panelini kolayca kullanma ve yönetme imkanı sağlayacak bir altyapı sunmayı amaçlamaktadır.Detaylı mimari ve metodolojik açıklamalar, "Metodoloji" bölümünde bulunacaktır.

